\documentclass{beamer}

\usetheme{intridea}
\usepackage[lithuanian]{babel}
\usepackage[utf8x]{inputenc}
\def\LTfontencoding{L7x}
\PrerenderUnicode{ąčęėįšųūž}
\usepackage{times}
\usepackage[\LTfontencoding]{fontenc}
\usepackage[T1]{fontenc}
\usepackage{color}
\usepackage{verbatim}
\usepackage{wrapfig}
\usepackage{fancyvrb}
\usepackage{bm}
\usepackage{amsfonts}
\usepackage{float}
\usepackage{hyperref}
\usepackage{pgf}

\usepackage{LTPlius}
\usepackage{graphicx}
\usepackage{caption}
\usepackage{subfig}
\usepackage{anyfontsize}

\usefonttheme{professionalfonts}
\usefonttheme{serif}
\setbeamerfont{note page}{family*=pplx,size=\footnotesize} % Palatino for notes

\include{pythonlisting}

\title{Lipšico modeliais pagrįsti daugiakriterės optimizacijos algoritmai}
\subtitle{Ataskaita už I-ąjį doktorantūros kursą}
\author{\scriptsize \hspace{2em} Doktorantas: \hspace{2em} \small Albertas Gimbutas\\
\scriptsize \hspace{8em} Vadovas: \hspace{2em} \small prof. habil. dr. Antanas Žilinskas}
\institute[]
{\scriptsize \hspace{8em} Doktorantūros pradžios ir pabaigos metai:\hspace{2em} \small 2013 - 2017}
\date{\small \the\year-\the\month-\the\day}


\begin{document}

\begin{frame}
  \titlepage
\end{frame}


\begin{frame}[fragile]{Tyrimo objektas}
    \begin{itemize}
        \item Lipšico modeliais pagrįsti globalios optimizacijos algoritmai
            \medskip
        \item Daugiakriterės optimizacijos algoritmai
    \end{itemize}
\end{frame}


\begin{frame}[fragile]{Tyrimo tikslas}
    \begin{itemize}
        \item Sukurti ir ištirti Lipšico modeliais pagrįstą daugiakriterės
            optimizacijos metodą, kuris leistų tiksliau aproksimuoti Pareto frontą
    \end{itemize}
\end{frame}

\begin{frame}[fragile]{Planuojami rezultatai}
    \begin{enumerate}
        \item Pasiūlytas ir ištirtas Lipšico modeliais pagrįstas daugiakriterės optimizacijos metodas
        \item Pasiūlytojo metodo konvergavimo greičio įvertinimas
        \item Pasiūlytojo metodo realizacija
        \item Eksperimentinis pasiūlytojo metodo palyginimas su kitais metodais
    \end{enumerate}
\end{frame}


\begin{frame}[fragile]{2013/2014m. darbo planas}
    \begin{enumerate}
    \item Išlaikyti egzaminus: 
    \begin{itemize}
        \item \textbf{Lygiagretieji ir paskirstytieji skaičiavimai}
            % \begin{itemize}
            %     \item \scriptsize \textit{Lygiagretieji algoritmai ir tinklinės
            %         technologijos}. R. Čiegis. Vilnius, Technika. 2005.
            % \end{itemize}
        \item \textbf{Daugiamačių duomenų vizualizavimo metodai}
            % \begin{itemize}
            %     \item \scriptsize \textit{Daugiamačių duomenų vizualizavimo
            %         metodai}. G. Dzemyda, O. Kurasova, J. Žilinskas. Vilnius, Mokslo aidai. 2008.
            % \end{itemize}
        \item \textbf{Globaliojo optimizavimo metodai}
            % \begin{itemize}
            %     \item \scriptsize \textit{Global Optimization}. A. T\"{o}rn. A.
            %         Žilinskas. New York, Springer-Verlag. 1989.
            % \end{itemize}
        \item \textbf{Optimizacijos teorija ir algoritmų sudėtingumas}
            % \begin{itemize}
            %     \item \scriptsize \textit{Iterative Methods for Optimization}.
            %         C. T. Kelley. Philadelphia, SIAM. 1999.
            %     \item \scriptsize \textit{Introduction to Algorithms, 3rd edition}. T.
            %         Cormen, C. Leiserson, R. Rivest, C. Stein. London, The MIT
            %         Press. 2009. Skyreliai: \textit{ 34. NP-Completeness, 35.
            %         Approximation Algorithms}.
            % \end{itemize}
    \end{itemize}
    % \hskip 0.4cm \footnotesize \textit{Visi egzaminai išlaikyti ir įvertinti 10.} \normalsize

    \bigskip
    \item  Apžvelgti ir išanalizuoti daugiakriterės optimizacijos algoritmus.
\end{enumerate}
\end{frame}


\begin{frame}[fragile]{\LARGE Ataskaita už 2013/2014m. mokslo metus}
    \begin{enumerate}
        \item Išlaikyti visi egzaminai. Visi įvertinti 10 balų.
        \smallskip
        \item Apžvelgti ir išanalizuoti daugiakriterės optimizacijos
            algoritmai. \footnotesize \textit{Atlikta ruošiantis 3. Globaliojo optimizavimo metodai egzaminui.} \normalsize
        \item Dalyvauta 2 savaičių mokymuose \textit{"Euro PhD on MCDM"}, Ispanijoje.
        \smallskip
        \item Pradėtas rengti straipsnio rankraštis. \\
            \begin{itemize}
        \footnotesize \item Pasiūlytas naujas metodas dviejų dimensijų dviejų tikslo funkcijų Lipšico
            optimizavimo probelmoms spręsti.\normalsize
        \end{itemize}
    \end{enumerate}
\end{frame}

\begin{frame}{2014/2015 m. darbo planas}
    \begin{enumerate}
        \item Apžvelgti ir išanalizuoti Lipšico modeliais pagrįstus
            daugiakriterės optimizacijos algoritmus
        \item Parengti disertacijos apžvalginės dalies pradinę versiją
        \item Sudaryti ir aprašyti tyrimo metodiką
    \end{enumerate}
\end{frame}

\end{document}
